\documentclass[./main.tex]{subfiles}

\begin{document}

\section{Problem description}
There are $N$ warehouses on different locations, each warehouse $w$ has a set-up price $s_w$ and a capacity $cap_w$.
Also, there are $M$ customers and each customer $c$ damands $d_c$ goods.
There is defined delivery cost $t_{cw}$ between each pair of customer and warehouse.

The goal is to assign one warehouse to each customer, while satisfying their demands and minimizing the overall cost.

\begin{equation}
    \begin{array}{ll @{} ll}
        \text{minimize}   & f = \displaystyle\sum_{w = 1}^{N}(I(|a_w| > 0) \cdot s_w + \sum_{c \in a_w} t_{cw}) & \\
        \text{subject to} & \displaystyle\sum_{c \in a_w} d_c \leq cap_w & \forall w = 1, \dots, N \\
                          & \displaystyle\sum_{w = 1}^{N} I(c \in a_w) = 1 & \forall c = 1, \dots, M \\
    \end{array}
\end{equation}

\begin{table}[ht]
    \centering
    \begin{tabular}{c | c}
        problem & optimum value \\
        \hline
        wl\_16\_1 & 976738.625 \\
        wl\_25\_2 & 796648.438 \\
        wl\_50\_1 & 793439.562 \\
        wl\_100\_4 & 17765201.949 \\
        wl\_200\_1 & 2686.479 \\
        wl\_500\_1 & 2608.148 \\
        wl\_1000\_1 & 5283.757 \\
        wl\_2000\_1 & 10069.803 \\
    \end{tabular}
    \caption{Problem instances}
\end{table}

\section{Representation and fitness function}

\textbf{Candidate representation}

Customers are represented as numbers $c = 1, 2, \dots, M$ and warehouses as numbers $w = 1, 2, \dots, N$.
Each candidate solution is represented as an integer vector of length $M$.
On each position $c$ is a number of warehouse $w$ which was assigned to customer $c$.

\vspace{0.5cm}
\begin{center}
    \begin{tabular}{| c | c | c | c | c |}
        \multicolumn{1}{c}{$1$} & \multicolumn{1}{c}{$2$} & \multicolumn{1}{c}{$\cdots$} & \multicolumn{1}{c}{$M - 1$} & \multicolumn{1}{c}{$M$} \\
        \hline
        $w_1$ & $w_2$ & $\cdots$ & $w_{M-1}$ & $w_M$ \\
        \hline
    \end{tabular}
\end{center}
\vspace{0.5cm}

Each candidate solution also holds excesses of individual capacities as a vector of length $N$, also as fitness function and constraint violation sum.

\textbf{Fitness function}

Fitness function is split into two parts: overall cost and constraint violations.
The overall cost is computed according to problem fitness function as specified in previous section.
Constraint violations are computed as absolute value of difference between capacity and accumulated demand after assiging the warehouses to customers.
It is useful this way, because it allows to use two different comparison operators.

First operator compares two candidates by their constraint violations and prefers the less violating.
If both candidates are feasible, only then they are compared by their fitness e.g. overall cost.
This operator is used mostly in finding and comparing \textit{bsf} candidates.

Second operator uses weighted sum of these two properties. It is usually managed by multipling constraing violations with some large number ($1000$ in my case) and adding the result with fitness.
This operator showed a bit better results in evolution operators as it can keep some infeasible but otherwise really fit candidates in the population.

\end{document}

