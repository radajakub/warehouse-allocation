\documentclass[./main.tex]{subfiles}

\begin{document}
\subsection{Memetic algorithm}
Third algorithm is a memetic algorithm in which I tried to combine good things from local search and good things from evolution algorithms and get the best results.
Solver is implemented in \textit{src/specialized/specialized.cpp} and is run by binary \textit{out/specialized} made from main file \textit{src/solver/specialized.cpp}.

\subsubsection{Description}

A smaller population of only $20$ candidates is held by the algorithm.
This is mostly to reduce execution time and keep more evaluations for the local search part.
Those candidates are initialized randomly as in the basic evolution algorithm, but then one of them is replaced by a greedily initialized candidate as in the simple local search.
The main idea of this one candidate is to get some good "genes" into the population even before the evolution starts.

The breeding phase is very similar to the basic evolution algorithm.
First, $5$ parents are selected from old population by tournaments of size $4$.
Secondly, in the crossover phase, a uniform crossover is used instead of single point one.
This crossover selects an assignment from a parent with probability $0.5$.
Lastly, in the mutation phase, an offspring is either changed in random $5$ assignments (note, that the evolution algorithm mutated only $1$) with probability $0.2$, or is shuffled with probability $0.6$.
The mutation is more aggressive than in the evolution algorithm as an attempt to decrease stagnation of population and to explore larger space.
Also, the probabilies were adjusted to prefer shuffle over random perturbation.

After breeding phase, a local search is started for each solution with probability $0.3$.
The search lasts $100$ iterations and as perturbation, the candidate is either again changed in random $5$ assignments with probability $0.5$, or an assignment with highest cost is found and to this customer is assigned cheapest possible warehouse.
Idea behind this is, that it should try to push some candidates to cheaper state at the cost of possible infeasibility and thus introduce some different genes.
These improved candidates are added to the new population and they participate in evaluation and recombination.

Recombination is the same as in the evolution algorithm.

It can be said, that this algorithm performs well on all instances, given enough iterations.
However, on some easier ones is not as good as it could be, but on the larger instances, it gets really close to global optimum.
It is not too slow and it seems that it always finds a valid solution.

\end{document}
