\documentclass[a4paper, 12pt]{article}

\usepackage[utf8]{inputenc}
\usepackage{fancyhdr}
\usepackage[a4paper, total={18cm, 25cm}]{geometry}
\usepackage[parfill]{parskip}
\usepackage[ddmmyyyy]{datetime}
\usepackage{hyperref}
\usepackage{subfiles}
% math
\usepackage{amsfonts}
\usepackage{amsmath}
% images
\usepackage{graphicx}
\graphicspath{ {./images/} }

\usepackage[table]{xcolor}

\setlength{\headheight}{15pt}
% \renewcommand{\arraystretch}{1.5}

\pagestyle{fancy}
\fancyhf{}
\lhead{\today}
\chead{Warehouse allocation - report}
\rhead{Jakub Rada (radajak5)}
\cfoot{\thepage}

\title{Warehouse allocation}
\author{Jakub Rada}
\date{}

\begin{document}

\maketitle
\pagebreak

\subfile{./definition.tex}

\section{Algorithms and experiments}

All algorithms can be terminated either by number of calls of function \textit{Candidate.evaluate()}, or by number of unimproving iterations / generations.
The submitted version uses the first one and the limit number is set by global object \textit{state.set\_limit(n)}.
Note, that the summary will print out slightly larger number of evaluations than the limit, but it is only caused by checking the condition after evaluating the whole population for more simplicity.
The \textit{bsf} solution is not influenced by these evaluations over the limit.

To change the termination condition to unimproving generations, it suffices to remove lines containing \textit{state.enough()}.
Then, the \textit{state.set\_limit(n)} corresponds to iterations in local search and to generations in evolution algorithm and specialized algorithm.

\subfile{./ls.tex}

\newpage

\subfile{./ea.tex}

\newpage

\subfile{./specialized.tex}

\newpage

\subfile{./comparison.tex}

\newpage

\subfile{./conclusion.tex}

\end{document}

