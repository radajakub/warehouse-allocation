\documentclass[./main.tex]{subfiles}

\begin{document}
\subsection{Evolution algorithm}
Second algorithm is basic evolution algorithm.
Solver is implemented in \textit{src/ea/ea.cpp} and is run by binary \textit{out/ea} made from main file \textit{src/solver/ls.cpp}.

\subsubsection{Description}

Algorithms keeps a population of 50 candidate solutions (mainly due to performance reasons, evaluating larger population is too slow).
Each of this solutions is randomly initialized.

In each iteration are chosen $5$ candidates as parents for crossover using tournament selection.
Size of each tournament is $4$ candidates and are compared by weighted sum of their objective functions and constraint violations, while penalty constant being $1000$.

Each pair of the $5$ selected parents is bred using single point crossover at randomly selected point and so for each pair of parents there is a new pair of offsprings.

After crossover, there are actually two mutation operators each with different probability.
First operator is the simple mutation from local search, where one random candidate is randomly assigned some warehouse.
This operator is chosen with probability \textit{Settings.mutation\_prob} $ = 0.5$.

When the first operator is not selected for mutation, the second operator is used with probability \textit{Settings.shuffle\_prob} $ = 0.3$.
This operator randomly shuffles the chromosome and has shown to be powerful in exploring new parts of search space.

After evaluation and update of \textit{bsf} solution, the new population is recombined with the old one.
All candidates from the new population are kept and then candidates from the old one are chosed with the same tournament procedure as in selection phase to get the recombined population to $50$ members.

When algorithm detects number of evaluations equal or higher to the limit, it does not update best so far solution and terminates.

From the experiments, this algorithm is really slow.
On the easier instances, it is necessary to make many evaluations and generations to come closer to global optimum.
However, on the more difficult instances, it starts to get closer, even though it takes very long.
Also, all solutions are feasible.
It is quite possible, that the population starts to stagnate as it's progressing and rely only on mutation and that it is not enough to search fast enough.

\end{document}
